\documentclass[10pt]{beamer}
\usetheme{Singapore}
\usecolortheme{default}
\usecolortheme{orchid}
\useoutertheme{infolines}
\useinnertheme[shadow=true]{rounded}

\newcommand\smallfont{\fontsize{8pt}{7.2}\selectfont}

\usepackage{multimedia}

\title{Introduction to Anaconda}
\titlegraphic{\includegraphics[height=2.0cm]{../logo.png}}
\logo{\includegraphics[height=0.5cm]{../logo.png}}
\author{{Stephen Weston} \and {Robert Bjornson}}
\institute[Yale]{
  Yale Center for Research Computing \\
  Yale University
}
\date{April 2016}
\begin{document}


%----------- titlepage ----------------------------------------------%
\begin{frame}[plain]
  \titlepage
\end{frame}

%----------- slide --------------------------------------------------%
\begin{frame}
\frametitle{Why Anaconda?}

\begin{itemize}
\item It comes with many packages for science and data analysis
\item NumPy, SciPy, Matplotlib, Pandas, IPython, Cython
\item Includes ``Spyder'', a Python development environment
\item Non-Python packages: curl, gcc, libboost
\item Available for Mac, Windows and Linux
\item Includes ``conda'', a platform independent package manager
\item Compatible with pip and PyPI
\end{itemize}
\end{frame}

%----------- slide --------------------------------------------------%
\begin{frame}
\frametitle{Installing Anaconda}

\begin{itemize}
\item Download installer from \url{https://www.continuum.io/download}
\item Execute the installer and follow the instructions
\end{itemize}
\end{frame}

%----------- slide --------------------------------------------------%
\begin{frame}
\frametitle{Introduction to Conda}
\begin{itemize}
\item Platform-independent package manager
\item Doesn't require administrative privilege to install packages
\item Provides "virtual environment" capabilities
\item Many channels exist that support additional packages
\item Installed on Yale clusters
\end{itemize}
\end{frame}

%----------- slide --------------------------------------------------%
\begin{frame}[fragile]
\frametitle{Installing Python packages with Conda}
It can be as easy as:
\begin{verbatim}
$ conda install wxpython
\end{verbatim}
Some packages are only available in special channels:
\begin{verbatim}
$ conda install -c vpython vpython
\end{verbatim}
Specific versions of packages can be requested:
\begin{verbatim}
$ conda install wxpython=3.0
\end{verbatim}
\end{frame}

%----------- slide --------------------------------------------------%
\begin{frame}[fragile]
\frametitle{Virtual Environments}
Environments allow different versions of packages
\begin{verbatim}
$ conda create --name test numpy=1.7
\end{verbatim}
On Mac OS X and Linux:
\begin{verbatim}
$ source activate test
$ python
...
$ source deactivate
\end{verbatim}
On Windows:
\begin{verbatim}
$ activate test
$ python
...
$ deactivate
\end{verbatim}
\end{frame}

%----------- slide --------------------------------------------------%
\begin{frame}[fragile]
\frametitle{More Conda operations}
Display information about all installed packages:
\begin{verbatim}
$ conda list
\end{verbatim}
Display information about available packages:
\begin{verbatim}
$ conda search '^doc'  # packages that start with "doc"
\end{verbatim}
Display outdated packages:
\begin{verbatim}
$ conda search --outdated
\end{verbatim}
Update a package:
\begin{verbatim}
$ conda update wxpython
\end{verbatim}
Uninstall a package:
\begin{verbatim}
$ conda remove wxpython
\end{verbatim}
\end{frame}

\end{document}
