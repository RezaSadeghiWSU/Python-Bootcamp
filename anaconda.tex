\documentclass[10pt]{beamer}
\usetheme{Singapore}
\usecolortheme{default}
\usecolortheme{orchid}
\useoutertheme{infolines}
\useinnertheme[shadow=true]{rounded}

\newcommand\smallfont{\fontsize{8pt}{7.2}\selectfont}

\usepackage{multimedia}

\title{Introduction to Anaconda}
\titlegraphic{\includegraphics[height=2.0cm]{../logo.png}}
\logo{\includegraphics[height=0.5cm]{../logo.png}}
\author{{Stephen Weston} \and {Robert Bjornson}}
\institute[Yale]{
  Yale Center for Research Computing \\
  Yale University
}
\date{August 2016}
\begin{document}


%----------- titlepage ----------------------------------------------%
\begin{frame}[plain]
  \titlepage
\end{frame}

%----------- slide --------------------------------------------------%
\begin{frame}
\frametitle{What is Anaconda?}
Anaconda is a Python distribution that is particularly popular
for data analysis and scientific computing
\begin{itemize}
\item Open source project developed by Continuum Analytics, Inc.
\item Available for Windows, Mac OS X and Linux
\item Includes many popular packages:
  NumPy, SciPy, Matplotlib, Pandas, IPython, Cython
\item Includes \textbf{Spyder}, a Python development environment
\item Includes \textbf{conda}, a platform-independent package manager
\end{itemize}
\end{frame}

%----------- slide --------------------------------------------------%
\begin{frame}[fragile]
\frametitle{Installing Anaconda}
Anaconda is easy to install
\begin{itemize}
\item Download installer from \url{https://www.continuum.io/download}
\item Execute the installer and follow the instructions
\end{itemize}
\vskip10pt
Anaconda is installed on Yale clusters Omega and Grace
\begin{verbatim}
$ module load Langs/Python/2.7-anaconda
\end{verbatim}
\end{frame}

%----------- slide --------------------------------------------------%
\begin{frame}
\frametitle{Introduction to Conda}
Simplifies installation of Python packages
\begin{itemize}
\item Platform-independent package manager
\item Doesn't require administrative privileges
\item Installs non-Python library dependencies (MKL, HDF5, Boost)
\item Provides "virtual environment" capabilities
\item Many \textit{channels} exist that support additional packages
\item Documentation at \url{http://conda.pydata.org.docs/}
\end{itemize}
\end{frame}

%----------- slide --------------------------------------------------%
\begin{frame}[fragile]
\frametitle{Installing Python packages with Conda}
It can be as easy as:
\begin{verbatim}
$ conda install wxpython
\end{verbatim}
Specific versions of packages can be requested:
\begin{verbatim}
$ conda install wxpython=3.0
\end{verbatim}
Some packages are only available in special channels:
\begin{verbatim}
$ conda install -c vpython vpython
\end{verbatim}
Pip can also be used:
\begin{verbatim}
$ pip install intervaltree
\end{verbatim}
\end{frame}

%----------- slide --------------------------------------------------%
\begin{frame}[fragile]
\frametitle{Virtual Environments}
Environments allow different versions of packages on same machine
\vskip10pt
Create environment ``test'' and install numpy version 1.7
\begin{verbatim}
$ conda create --name test numpy=1.7
\end{verbatim}
Environments must be activated
\vskip10pt
On Mac OS X and Linux:
\begin{verbatim}
$ source activate test
$ python
...
$ source deactivate
\end{verbatim}
On Windows:
\begin{verbatim}
> activate test
> python
...
> deactivate
\end{verbatim}
\end{frame}

%----------- slide --------------------------------------------------%
\begin{frame}[fragile]
\frametitle{More Conda operations}
Display help for conda command (and sub-commands):
\begin{verbatim}
$ conda --help
$ conda list --help
\end{verbatim}
List packages in current conda environment:
\begin{verbatim}
$ conda list
\end{verbatim}
Display all information about conda installation:
\begin{verbatim}
$ conda info -a
\end{verbatim}
Search for available packages (using regular expressions):
\begin{verbatim}
$ conda search '^doc'  # packages that start with "doc"
\end{verbatim}
Update package:
\begin{verbatim}
$ conda update wxpython
\end{verbatim}
Uninstall package:
\begin{verbatim}
$ conda remove wxpython
\end{verbatim}
\end{frame}

%----------- slide --------------------------------------------------%
\begin{frame}[fragile]
\frametitle{Warning about installing/updating packages}
Installing packages into existing environment may cause problems
\begin{itemize}
\item best to install all necessary packages at same time
\item \textbf{install} command may ask permission to update existing packages
which may result in version conflicts
\item install packages into new environments if possible to avoid conflicts
\end{itemize}
\end{frame}

\end{document}
