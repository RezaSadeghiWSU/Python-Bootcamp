\documentclass[10pt]{beamer}
\usetheme{Singapore}
\usecolortheme{default}
\usecolortheme{orchid}
\useoutertheme{infolines}
\useinnertheme[shadow=true]{rounded}

\usepackage{multimedia}

\title{Introduction to Python Data Analysis}
\titlegraphic{\includegraphics[height=3.0cm]{../logo.png}}
\author{{Stephen Weston} \and {Robert Bjornson}}
\institute[Yale]{
  Yale Center for Research Computing \\
  Yale University
}
\date{July 2016}
\begin{document}

%----------- titlepage ----------------------------------------------%
\begin{frame}[plain]
  \titlepage
\end{frame}

%----------- slide --------------------------------------------------%
\begin{frame}
\frametitle{Python for data analysis}
Python is more of a general purpose programming language than
R or Matlab.
It has gradually become more popular for data analysis and
scientific computing, but additional modules are needed.
Some of the more popular modules are:
\begin{description}
\item[NumPy] N-dimensional array
\item[SciPy] Scientific computing (linear algebra, numerical integration, optimization, etc)
\item[Matplotlib] 2D Plotting (similar to Matlab)
\item[IPython] Enhanced Interactive Console
\item[Sympy] Symbolic mathematics
\item[Pandas] Data analysis (provides a data frame structure similar to R)
\end{description}
\vskip10pt
NumPy, SciPy and Matplotlib are used in this presentation.
\end{frame}

%----------- slide --------------------------------------------------%
\begin{frame}[fragile]
\frametitle{Creating N-dimensional arrays using NumPy}
There are many ways to create N-dimensional arrays
\begin{verbatim}
import numpy as np
# Create 2X3 double precision array initialized to all zeroes
a = np.zeros((2,3), dtype=np.float64)

# Create array initialized by list of lists
a = np.array([[0,1,2],[3,4,5]], dtype=np.float64)

# Create array by reading CSV file
a = np.genfromtxt('data.csv', dtype=np.float64, delimiter=',')

# Create array using "arange" function
a = np.arange(6, dtype=np.float64).reshape(2,3)
\end{verbatim}
\end{frame}

%----------- slide --------------------------------------------------%
\begin{frame}[fragile]
\frametitle{Get values from N-dimensional array}
NumPy provides many ways to extract data from arrays
\begin{verbatim}
# Print single element of 2D array
print a[0,0]      # a scalar, not an array

# Print first row of 2D array
print a[0,:]      # 1D array

# Print last column of array
print a[:,-1]     # 1D array

# Print sub-matrix of 2D array
print a[0:2,1:3]  # 2D array

\end{verbatim}
\end{frame}

%----------- slide --------------------------------------------------%
\begin{frame}[fragile]
\frametitle{Modifying N-dimensional arrays}
NumPy uses the same basic syntax for modifying arrays
\begin{verbatim}
# Assign single value to single element of 2D array
a[0,0] = 25.0

# Assign 1D array to first row of 2D array
a[0,:] = np.array([10,11,12], dtype=np.float64)

# Assign 1D array to last column of 2D array
a[:,-1] = np.array([20,21], dtype=np.float64)

# Assign 2D array to sub-matrix of 2D array
a[0:2,1:3] = np.array([[10,11],[20,21]], dtype=np.float64)
\end{verbatim}
\end{frame}

%----------- slide --------------------------------------------------%
\begin{frame}[fragile]
\frametitle{Modifying arrays using broadcasting}
\begin{verbatim}
# Assign scalar to first row of 2D array
a[0,:] = 10.0

# Assign 1D array to all rows of 2D array
a[:,:] = np.array([30,31,32], dtype=np.float64)

# Assign 1D array to all columns of 2D array
a[:,:] = np.array([40,41], dtype=np.float64).reshape(2,1)

# Assign scalar to sub-matrix of 2D array
a[0:2,1:3] = 100.0
\end{verbatim}
\end{frame}

%----------- slide --------------------------------------------------%
\begin{frame}[fragile]
\frametitle{Arithmetic on arrays}
Operate on arrays using binary operators and NumPy functions
\begin{verbatim}
# Create 1D array
a = np.arange(4, dtype=np.float64)

# Add 1D arrays elementwise
a + a

# Multiply 1D arrays elementwise
a * a

# Sum elements of 1D array
a.sum()

# Compute dot product
np.dot(a, a)    # same as: (a * a).sum()

# Compute cross product
np.dot(a.reshape(4,1), a.reshape(1,4))
\end{verbatim}
\end{frame}

%----------- slide --------------------------------------------------%
\begin{frame}[fragile]
\frametitle{SciPy Linear Algebra functions}
\begin{verbatim}
import numpy as np
from scipy import linalg
a = np.array([[1, 2], [3, 4]], dtype=np.float64)

# Compute the inverse matrix
linalg.inv(a)

# Compute singular value decomposition
linalg.svd(a)

# Compute eigenvalues
linalg.eigvals(a)
\end{verbatim}
\end{frame}

%----------- slide --------------------------------------------------%
\begin{frame}[fragile]
\frametitle{2D plotting using Matplotlib}
\begin{verbatim}
import numpy as np
import matplotlib.pyplot as plt
x = np.linspace(0.0, 2.0, 20)

plt.plot(x, np.sqrt(x), 'ro')  # red circles
plt.show()

plt.plot(x, np.sqrt(x), 'b-')  # blue lines
plt.show()

# Three plots in one figure
plt.plot(x, x, 'g--', x, np.sqrt(x), 'ro', x, np.sqrt(x), 'b-')
plt.show()
\end{verbatim}
\end{frame}

\end{document}
